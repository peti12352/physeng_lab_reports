\documentclass[12pt,a4paper]{article}
\usepackage{graphicx, booktabs, siunitx, amsmath, geometry, float}
\usepackage{subcaption}
\geometry{margin=2.5cm}
\title{XXX}
\author{Gregorio Jaca U8L9B9, Peter Tallosy K14WR1 }
\date{XXX, 2025}

\begin{document}
\maketitle

% Abstract
\begin{abstract}

    Sensitive light polarization allows us to characterize the optical properties of different samples.
    We measured the optical retardation of a voltage controlled liquid cristal (LC) cell. We built a setup which allows us to measure the linear dichroism in hemozoid in suspension, a malaria indicator. Using samples of known concentration, we constructed a calibration curve which allows us to quantify the concentration of an unknown sample.

\end{abstract}

% Introduction
\section{Introduction}



\section{Results and Discussion}

\subsection{Basic characterization of film polarizers}

\subsection{Liquid crystal cell}

Molecules with anisotropic shapes can have anisotropic orientation-dependent optical effects individually. In most cases, aggrgates of randomly oriented molecules have no net anisotropic effect. Liquid crystal (LC) phases at the boundary of a liquid and a solid crystal with an ordered orientation, which can controlled by an electric field. Taking advantage of this propery, we use a LC cell with a volatge controlled waveplate which generates the alternating electric field, which acts as a voltage-controlled waveplate.

\begin{figure}[H]
    \centering
    \begin{subfigure}{\linewidth}
        \centering
        \includegraphics[width=0.8\linewidth]{XXX.PNG}
    \end{subfigure}
    
    \vspace{1em} % 

    \begin{subfigure}{\linewidth}
        \centering
        \includegraphics[width=0.8\linewidth]{XXX.PNG}
    \end{subfigure}
    
    \caption{XXX}
    \label{fig:XXX}
\end{figure}

\subsection{Detection of hemozoin suspended in water}

% The results from the lockin measurement were not very precise, and repeated mesurements yielded slightly different results. 

The quality of the calibration curve is limited by the few datapoints available. We were expecting a linear relationship, which is difficult to validate given the limited data. The discprenacy seen at the 0\% concentration sample could be accounted for by considering it the baseline effect caused by the elements other than the hemozoid. % XXX improved
Furthermore, there were more bubbles in the 0\% concentration sample. However, linear or nonlinear, a well done calibration curve can be effectively used to calculate the concentration level of hemozoid by interpolation, which makes this technique useful for diagnosis.
% G: idk if we should expect linear relationship. the handout doesnt suggest anything


% Conclusion
\section{Conclusion}


We could construct a calibration curve that relates the concentration of hemozoid to the measured optical signal. I would be greatly improved by increasing the number of datapoints used for the curve.

\end{document}