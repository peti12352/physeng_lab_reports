\documentclass[12pt,a4paper]{article}
\usepackage{graphicx, booktabs, siunitx, amsmath, geometry, float}
\usepackage{subcaption}
\geometry{margin=2.5cm}
\title{M1 & M2: Investigation of Transistor Amplifiers}
\author{Gregorio Jaca U8L9B9, Peter Tallosy K14WR1 }
\date{\today}

\begin{document}
\maketitle

\section{M1: Investigation of a Common-Emitter Amplifier}

\subsection{DC Operating Point}
The DC voltages of the amplifier were measured at the transistor's electrodes with the input left open.
\begin{itemize}
    \item $U_B = \SI{1.263}{V}$
    \item $U_C = \SI{7.42}{V}$
    \item $U_E = \SI{0.619}{V}$
\end{itemize}
From these, the base-emitter and collector-emitter voltages were calculated:
\begin{itemize}
    \item $U_{BE} = U_B - U_E = \SI{0.644}{V}$
    \item $U_{CE} = U_C - U_E = \SI{6.801}{V}$
\end{itemize}

\subsection{Saturation Measurement}
A \SI{1}{kHz} sinusoidal input signal was applied. The input amplitude was increased until the output signal began to show distortion. Saturation was observed to begin at an input amplitude of \SI{208}{mV}.

\subsection{Frequency Response}
The frequency response of the amplifier was measured by applying a sinusoidal input of constant amplitude and varying the frequency. The input and output voltages were recorded at several frequencies.

\begin{table}[H]
    \centering
    \caption{Frequency response data for Common-Emitter amplifier (M1.c).}
    \label{tab:freq_response_m1}
    \sisetup{table-format=3.2}
    \begin{tabular}{
        S[table-format=6.0]
        S[table-format=3.0]
        S[table-format=4.0]
        S[table-format=2.2]
        S[table-format=2.2]
        S[table-format=3.0]
    }
        \toprule
        {$f$ (\si{Hz})} & {$U_{in}$ (\si{mV})} & {$U_{out}$ (\si{mV})} & {$A_u$} & {$a_u$ (\si{dB})} & {$\phi$ (\si{\degree})} \\
        \midrule
        50 & 120 & 4800 & 40.00 & 32.04 & 108 \\
        100 & 118 & 6400 & 54.24 & 34.68 & 134 \\
        200 & 112 & 7200 & 64.29 & 36.16 & 147 \\
        500 & 112 & 7400 & 66.07 & 36.40 & 160 \\
        1000 & 110 & 7400 & 67.27 & 36.56 & 164 \\
        2000 & 110 & 7400 & 67.27 & 36.56 & 166 \\
        5000 & 112 & 7400 & 66.07 & 36.40 & 170 \\
        10000 & 112 & 7400 & 66.07 & 36.40 & 178 \\
        20000 & 110 & 7200 & 65.45 & 36.32 & 187 \\
        50000 & 110 & 6200 & 56.36 & 35.02 & 208 \\
        100000 & 112 & 4800 & 42.86 & 32.64 & 225 \\
        \bottomrule
    \end{tabular}
\end{table}

\begin{figure}[H]
    \centering
    % GGG: Create Bode plot (au vs f) and phase plot (phi vs f) from the table data for M1.
    \includegraphics[width=0.7\linewidth]{bode_plot_m1.png}
    \caption{Bode plot of the common-emitter amplifier's frequency response.}
    \label{fig:bode_plot_m1}
\end{figure}

\subsection{Cut-off Frequencies and Bandwidth}
The lower and upper cut-off frequencies were determined to be $f_l = \SI{57}{Hz}$ and $f_u = \SI{82.3}{kHz}$. The bandwidth of the amplifier is $B = f_u - f_l = \SI{82.24}{kHz}$.

\subsection{Miller Effect}
A \SI{22}{pF} capacitor was connected between the collector and base of the transistor. This reduced the upper cut-off frequency to $f_u = \SI{12}{kHz}$.

\subsection{Cascade Circuit}
The circuit was modified into a cascade configuration. This increased the upper cut-off frequency significantly to $f_u = \SI{271}{kHz}$.

\section{M2: Investigation of a Common-Collector Amplifier}

\subsection{Gain and Phase Shift}
At \SI{1}{kHz}, with an input of $U_{in} = \SI{1010}{mV}$, the output was $U_{out} = \SI{920}{mV}$. This gives a voltage gain of $A_u = 0.91$. The phase shift was $\phi = \SI{0.2}{\degree}$.

\subsection{Frequency Response}
The frequency response was investigated at several input frequencies.

\begin{table}[H]
    \centering
    \caption{Frequency response data for Common-Collector amplifier (M2.b).}
    \label{tab:freq_response_m2}
    \sisetup{table-format=1.3}
    \begin{tabular}{
        S[table-format=7.0]
        S[table-format=4.0]
        S[table-format=3.0]
        S[table-format=1.3]
        S[table-format=1.1]
    }
        \toprule
        {$f$ (\si{Hz})} & {$U_{in}$ (\si{mV})} & {$U_{out}$ (\si{mV})} & {$A_u$} & {$\phi$ (\si{\degree})} \\
        \midrule
        100 & 1050 & 960 & 0.914 & 0.0 \\
        1000 & 1010 & 920 & 0.911 & 0.2 \\
        10000 & 1010 & 920 & 0.911 & 0.0 \\
        100000 & 1010 & 928 & 0.919 & 0.0 \\
        1000000 & 1000 & 928 & 0.928 & 7.2 \\
        \bottomrule
    \end{tabular}
\end{table}

\subsection{Input Resistance}
The input resistance of the common-collector amplifier was measured to be $R_{in} = \SI{37}{k\ohm}$.

\subsection{High Input Resistance Configuration}
The circuit was modified to a high input resistance configuration. The input resistance was measured to be $R_{in} = \SI{151}{k\ohm}$.

\section{Conclusion}
The characteristics of a common-emitter amplifier were measured. The operating point, gain, and bandwidth were determined. The measurements confirm the expected behavior, including the \SI{180}{\degree} phase inversion and the frequency limitations of the amplifier. The Miller effect and the cascade configuration demonstrated methods to influence the high-frequency performance.

The common-collector amplifier was also investigated, confirming its characteristics such as voltage gain close to unity, no phase inversion, and high input impedance.

\end{document}
